\documentclass[12pt,a4paper]{article}


\usepackage[utf8]{inputenc}
\usepackage[T1]{fontenc}
\usepackage[french]{babel}
\usepackage{geometry}
\usepackage{graphicx}
\usepackage{float}
\usepackage{amsmath}
\usepackage{booktabs}
\usepackage{longtable}
\usepackage{hyperref}
\usepackage{setspace}

\geometry{margin=2.5cm}
\onehalfspacing


\begin{document}


\begin{titlepage}
\centering

\textbf{République Algérienne Démocratique et Populaire}\\
\vspace{0.2cm}
\textbf{Ministère de l’Enseignement Supérieur}\\
\vspace{0.2cm}
\textbf{Université des Sciences et de la Technologie Houari Boumediene}\\
\vspace{0.2cm}
\textbf{Département d’Informatique}

\vspace{3cm}

{\huge \textbf{Manuel d’Utilisation}}\\[0.5cm]
{\Large Système de Gestion de Transport et de Livraison}

\vspace{3cm}

Rapport réalisé dans le cadre du module\\
\textit{Systèmes d’Information 2}

\vspace{3cm}
\begin{table}[H]
\centering
\large
\begin{tabular}{p{7cm} p{4cm}}
\toprule
\textbf{Membre} & \textbf{Matricule} \\
\midrule
Aissat Lyna & 232331406109 \\
Boussaha Sara Alaa & 232331406003\\
Hachi Kawthar Khadidja & 232331490809 \\
\midrule
\multicolumn{2}{l}{\textbf{Enseignant : RACHID BOUDOUR}} \\
\bottomrule
\end{tabular}
\end{table}
\vspace{2cm}
\textbf{Année Universitaire 2025--2026}

\end{titlepage}

\thispagestyle{empty}
\newpage


\tableofcontents
\newpage


\section{Introduction}
Ce manuel d’utilisation a pour objectif de guider l’utilisateur dans l’utilisation du système de gestion de transport et de livraison.  
Il présente les principales fonctionnalités de l’application ainsi que les étapes nécessaires pour effectuer les opérations courantes.

Ce document est destiné aux utilisateurs finaux du système.

\newpage

\section{Accès à l’Application et Interface Générale}
L’application est accessible via un navigateur web.L’utilisateur accède à l’interface principale , elle comprend : 
\begin{itemize}
    \item Une barre de navigation.
    \item Un tableau de bord.
    \item Des menus permettant l’accès aux différentes fonctionnalités.
\end{itemize}

\begin{figure}[H]  
    \centering
    \hspace*{-2.5cm}
    \includegraphics[width=1.3 \linewidth]{interface generale.png}
    \caption{interface Generale}
    \label{fig:placeholder}
\end{figure}

\newpage

\section{Gestion des Tables}
Le système permet la gestion de plusieurs tables de données

\subsection{Tables de Bases}
\begin{figure}[H]  
    \centering
    \hspace*{-2cm}
    \includegraphics[width=1\linewidth]{navigationtable.png}
    \caption{navigation}
    \label{fig:placeholder}
\end{figure}

\vspace{1cm}

\subsection{Expeditions, Tournee et Colis}
\begin{figure}[H]  
    \centering
    \hspace*{-2cm}
    \includegraphics[width=1\linewidth]{navigation2.png}
    \caption{navigation}
    \label{fig:placeholder}
\end{figure}

\vspace{2cm}
\subsection{Facturation et paiements}
\begin{figure}[H]  
    \centering
    \hspace*{-2cm}
    \includegraphics[width=1\linewidth]{navigation3.png}
    \caption{navigation}
    \label{fig:placeholder}
\end{figure}

\vspace{2cm}
\subsection{Reclamations et Incidents}
\begin{figure}[H]  
    \centering
    \hspace*{-2cm}
    \includegraphics[width=1\linewidth]{navigation4.png}
    \caption{navigation}
    \label{fig:placeholder}
\end{figure}

\vspace{5cm}

\subsection{Recherche et Filtrage}
\begin{itemize}
    \item Recherche d’un enregistrement via une barre de recherche.
    \item Filtrage des données selon critère OLD et NEW
\end{itemize}

\begin{figure}[H]  
    \centering
    \hspace*{-2cm}
    \includegraphics[width=1.2\linewidth]{recherche.png}
    \caption{Recherche}
    \label{fig:placeholder}
\end{figure}

\vspace{3cm}

\newpage
\subsection{Pagination}
Les données sont affichées page par page afin de faciliter la navigation.

\begin{figure}[H]  
    \centering
    \hspace*{-2cm}
    \includegraphics[width=1.3\linewidth]{pagination.png}
    \caption{Pagination}
    \label{fig:placeholder}
\end{figure}



\subsection{Actions sur les Enregistrements}
\begin{itemize}
    \item Ajout, Modification et supression d’un enregistrement
    \item Afficher les informations d'un enregistrement.
\end{itemize}

\begin{figure}[H]  
    \centering
     \hspace*{-2.2cm}
    \includegraphics[width=1.15\linewidth]{Ajout.png}
    \caption{Actions sur Enregistrements}
    \label{fig:placeholder}
\end{figure}



\section{Conclusion}
Ce manuel d’utilisation présente les fonctionnalités essentielles du système de gestion de transport et de livraison.  
Il permet à l’utilisateur de comprendre et d’utiliser efficacement l’application.

\end{document}
